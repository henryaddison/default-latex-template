\section{Introduction}

Climate change is predicted to cause intensification of heavy rainfall extremes in the UK \cite{donat2016precipextreme, kendon2019ukcpscience}. The aim of this project is to help understand impacts of climate change on UK rainfall at a local (\(\sim\) 2km) scale, including extreme events, to facilitate better adaptation decisions and inform mitigation policy.

Global climate models (GCMs) simulate physical processes of the climate. They are used to experiment with and explore the climate in different conditions such as different forcings due greenhouse gas (GHG) emission scenarios or paleoclimates of prehistoric Earth. The computational resource requirements of these models restrict their resolution to grids of 25km or more. This is too coarse to provide actionable insight \cite{gutierrez2019sdcomparison}.

The Met Office's UK Climate Projections introduced local projections at a resolution of 2.2km in the UKCP18 dataset \cite{ukcp18local} using a regional climate model (RCM), which still approximates the fundamental physical equations at a higher resolution by operating on a smaller spatial domain and using a GCM to set boundary conditions. However, these projections are a \textquote{major cost in terms of computing resource}\cite{kendon2019ukcpscience} and required a trade-off amongst length, domain size and ensemble size to produce. The projections are available for only 12 ensemble members over a single climate change scenario for three 20-year chunks, all driven by the same GCM \cite{kendon2019ukcpscience}. There are further technical challenges as well to implement such an RCM for a different driving GCM with a different codebase and programming interface.

Machine Learning (ML) techniques could emulate the more expensive high-resolution simulations using coarse GCM data as input and generate local projections more cheaply \cite{vandal2018mldownscaling, gentine2018mlsuperparam, rasp2018dlsubgridclimmodel}. Furthermore, by emulating an RCM rather than replicating historical observations, changing relationships in the climate might be learnable from the physics-based modelling of the GCM and RCM. This means projections from a climate model could be complemented with further samples with realistic spatial and temporal structure enabling better understanding of the uncertainty of high-resolution precipitation. This is particularly true for extreme events which are poorly sampled by their infrequent nature and the small ensemble size of the UKCP18 2.2km RCM.

Sections \ref{sec:downscalingbg} and \ref{sec:mlbg} cover the background to downscaling of GCMs and machine learning applications to related problems. The in-progress method and immediate next-steps for the ML-based emulator are described in Section \ref{sec:emulatorstudy}. Section \ref{sec:future} lays out possible future directions of work followed by a short conclusion in Section \ref{sec:conclusion}.
